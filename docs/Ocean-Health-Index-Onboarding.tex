% Options for packages loaded elsewhere
\PassOptionsToPackage{unicode}{hyperref}
\PassOptionsToPackage{hyphens}{url}
%
\documentclass[
]{book}
\usepackage{lmodern}
\usepackage{amssymb,amsmath}
\usepackage{ifxetex,ifluatex}
\ifnum 0\ifxetex 1\fi\ifluatex 1\fi=0 % if pdftex
  \usepackage[T1]{fontenc}
  \usepackage[utf8]{inputenc}
  \usepackage{textcomp} % provide euro and other symbols
\else % if luatex or xetex
  \usepackage{unicode-math}
  \defaultfontfeatures{Scale=MatchLowercase}
  \defaultfontfeatures[\rmfamily]{Ligatures=TeX,Scale=1}
\fi
% Use upquote if available, for straight quotes in verbatim environments
\IfFileExists{upquote.sty}{\usepackage{upquote}}{}
\IfFileExists{microtype.sty}{% use microtype if available
  \usepackage[]{microtype}
  \UseMicrotypeSet[protrusion]{basicmath} % disable protrusion for tt fonts
}{}
\makeatletter
\@ifundefined{KOMAClassName}{% if non-KOMA class
  \IfFileExists{parskip.sty}{%
    \usepackage{parskip}
  }{% else
    \setlength{\parindent}{0pt}
    \setlength{\parskip}{6pt plus 2pt minus 1pt}}
}{% if KOMA class
  \KOMAoptions{parskip=half}}
\makeatother
\usepackage{xcolor}
\IfFileExists{xurl.sty}{\usepackage{xurl}}{} % add URL line breaks if available
\IfFileExists{bookmark.sty}{\usepackage{bookmark}}{\usepackage{hyperref}}
\hypersetup{
  pdftitle={R for Excel Users},
  pdfauthor={Julie Lowndes \& Allison Horst},
  hidelinks,
  pdfcreator={LaTeX via pandoc}}
\urlstyle{same} % disable monospaced font for URLs
\usepackage{color}
\usepackage{fancyvrb}
\newcommand{\VerbBar}{|}
\newcommand{\VERB}{\Verb[commandchars=\\\{\}]}
\DefineVerbatimEnvironment{Highlighting}{Verbatim}{commandchars=\\\{\}}
% Add ',fontsize=\small' for more characters per line
\usepackage{framed}
\definecolor{shadecolor}{RGB}{248,248,248}
\newenvironment{Shaded}{\begin{snugshade}}{\end{snugshade}}
\newcommand{\AlertTok}[1]{\textcolor[rgb]{0.94,0.16,0.16}{#1}}
\newcommand{\AnnotationTok}[1]{\textcolor[rgb]{0.56,0.35,0.01}{\textbf{\textit{#1}}}}
\newcommand{\AttributeTok}[1]{\textcolor[rgb]{0.77,0.63,0.00}{#1}}
\newcommand{\BaseNTok}[1]{\textcolor[rgb]{0.00,0.00,0.81}{#1}}
\newcommand{\BuiltInTok}[1]{#1}
\newcommand{\CharTok}[1]{\textcolor[rgb]{0.31,0.60,0.02}{#1}}
\newcommand{\CommentTok}[1]{\textcolor[rgb]{0.56,0.35,0.01}{\textit{#1}}}
\newcommand{\CommentVarTok}[1]{\textcolor[rgb]{0.56,0.35,0.01}{\textbf{\textit{#1}}}}
\newcommand{\ConstantTok}[1]{\textcolor[rgb]{0.00,0.00,0.00}{#1}}
\newcommand{\ControlFlowTok}[1]{\textcolor[rgb]{0.13,0.29,0.53}{\textbf{#1}}}
\newcommand{\DataTypeTok}[1]{\textcolor[rgb]{0.13,0.29,0.53}{#1}}
\newcommand{\DecValTok}[1]{\textcolor[rgb]{0.00,0.00,0.81}{#1}}
\newcommand{\DocumentationTok}[1]{\textcolor[rgb]{0.56,0.35,0.01}{\textbf{\textit{#1}}}}
\newcommand{\ErrorTok}[1]{\textcolor[rgb]{0.64,0.00,0.00}{\textbf{#1}}}
\newcommand{\ExtensionTok}[1]{#1}
\newcommand{\FloatTok}[1]{\textcolor[rgb]{0.00,0.00,0.81}{#1}}
\newcommand{\FunctionTok}[1]{\textcolor[rgb]{0.00,0.00,0.00}{#1}}
\newcommand{\ImportTok}[1]{#1}
\newcommand{\InformationTok}[1]{\textcolor[rgb]{0.56,0.35,0.01}{\textbf{\textit{#1}}}}
\newcommand{\KeywordTok}[1]{\textcolor[rgb]{0.13,0.29,0.53}{\textbf{#1}}}
\newcommand{\NormalTok}[1]{#1}
\newcommand{\OperatorTok}[1]{\textcolor[rgb]{0.81,0.36,0.00}{\textbf{#1}}}
\newcommand{\OtherTok}[1]{\textcolor[rgb]{0.56,0.35,0.01}{#1}}
\newcommand{\PreprocessorTok}[1]{\textcolor[rgb]{0.56,0.35,0.01}{\textit{#1}}}
\newcommand{\RegionMarkerTok}[1]{#1}
\newcommand{\SpecialCharTok}[1]{\textcolor[rgb]{0.00,0.00,0.00}{#1}}
\newcommand{\SpecialStringTok}[1]{\textcolor[rgb]{0.31,0.60,0.02}{#1}}
\newcommand{\StringTok}[1]{\textcolor[rgb]{0.31,0.60,0.02}{#1}}
\newcommand{\VariableTok}[1]{\textcolor[rgb]{0.00,0.00,0.00}{#1}}
\newcommand{\VerbatimStringTok}[1]{\textcolor[rgb]{0.31,0.60,0.02}{#1}}
\newcommand{\WarningTok}[1]{\textcolor[rgb]{0.56,0.35,0.01}{\textbf{\textit{#1}}}}
\usepackage{longtable,booktabs}
% Correct order of tables after \paragraph or \subparagraph
\usepackage{etoolbox}
\makeatletter
\patchcmd\longtable{\par}{\if@noskipsec\mbox{}\fi\par}{}{}
\makeatother
% Allow footnotes in longtable head/foot
\IfFileExists{footnotehyper.sty}{\usepackage{footnotehyper}}{\usepackage{footnote}}
\makesavenoteenv{longtable}
\usepackage{graphicx}
\makeatletter
\def\maxwidth{\ifdim\Gin@nat@width>\linewidth\linewidth\else\Gin@nat@width\fi}
\def\maxheight{\ifdim\Gin@nat@height>\textheight\textheight\else\Gin@nat@height\fi}
\makeatother
% Scale images if necessary, so that they will not overflow the page
% margins by default, and it is still possible to overwrite the defaults
% using explicit options in \includegraphics[width, height, ...]{}
\setkeys{Gin}{width=\maxwidth,height=\maxheight,keepaspectratio}
% Set default figure placement to htbp
\makeatletter
\def\fps@figure{htbp}
\makeatother
\setlength{\emergencystretch}{3em} % prevent overfull lines
\providecommand{\tightlist}{%
  \setlength{\itemsep}{0pt}\setlength{\parskip}{0pt}}
\setcounter{secnumdepth}{5}
\usepackage{booktabs}
\usepackage[]{natbib}
\bibliographystyle{apalike}

\title{R for Excel Users}
\author{Julie Lowndes \& Allison Horst}
\date{2020-10-14}

\begin{document}
\maketitle

{
\setcounter{tocdepth}{1}
\tableofcontents
}
\hypertarget{welcome}{%
\chapter{Welcome}\label{welcome}}

\textbf{Deprecated draft matrials: please see \url{http://rstd.io/r-for-excel} for current teaching materials}

Hello! These are materials for a pilot workshop that was taught by Julie Stewart Lowndes and Allison Horst in preparation for the RStudio Conference in San Francisco January 27-28.

We are environmental scientists who use and teach R in our daily work. We both work at the University of California Santa Barbara: Julie is based at the National Center for Ecological Analysis and Synthesis as part of the Ocean Health Index team and leads Openscapes, and Allison is based at the Bren School of Environmental Science and Management as a lecturer of data science \& statistics --- and is also an Artist in Residence at RStudio.

\hypertarget{agenda-pilot-workshop}{%
\section{Agenda: pilot workshop}\label{agenda-pilot-workshop}}

\begin{longtable}[]{@{}lrr@{}}
\toprule
Time & Day 1 & Day 2\tabularnewline
\midrule
\endhead
9-10:30 & \protect\hyperlink{overview}{Motivation}, \protect\hyperlink{rstudio}{R \& RStudio, RMarkdown} & \protect\hyperlink{ggplot2}{Charts with \texttt{ggplot2}}\tabularnewline
break & &\tabularnewline
11-12:30 & \protect\hyperlink{github}{Intro to GitHub} & \protect\hyperlink{dplyr-pivot-tables}{\texttt{dplyr} \& Pivot Tables}\tabularnewline
lunch & &\tabularnewline
13:00-14:30 & \protect\hyperlink{readxl}{Importing data: \texttt{readxl}} & \protect\hyperlink{dplyr-vlookups}{\texttt{dplyr} \& VLOOKUPs}\tabularnewline
\bottomrule
\end{longtable}

From 14:30-15:00 each day we have time for additional help and feedback.

\hypertarget{prerequisites}{%
\section{Prerequisites}\label{prerequisites}}

Before the training, please make sure you have done the following:

\begin{enumerate}
\def\labelenumi{\arabic{enumi}.}
\tightlist
\item
  Download and install \textbf{up-to-date versions} of:

  \begin{itemize}
  \tightlist
  \item
    R: \url{https://cloud.r-project.org}
  \item
    RStudio: \url{http://www.rstudio.com/download}
  \end{itemize}
\item
  Install the Tidyverse
\item
  Create a an account:

  \begin{itemize}
  \tightlist
  \item
    \url{https://github.com}
  \end{itemize}
\end{enumerate}

\begin{enumerate}
\def\labelenumi{\arabic{enumi}.}
\tightlist
\item
  Get comfortable: if you're not in a physical workshop, be set up with two screens if possible. You will be following along in RStudio on your own computer while also watching a virtual training or following this tutorial on your own.
\end{enumerate}

\hypertarget{overview}{%
\chapter{Overview}\label{overview}}

\hypertarget{welcome}{%
\section{Welcome!}\label{welcome}}

In this workshop you will learn hands-on how to begin to interoperate between Excel and R. But this workshop is not only about learning R; we will learn R using additional software: RStudio and GitHub. These tools will help us develop good habits for working in a reproducible and collaborative way --- critical attributes of the modern analyst.

It's going to be fun and empowering!

\hypertarget{why-learn-r-if-i-know-excel}{%
\section{Why learn R if I know Excel?}\label{why-learn-r-if-i-know-excel}}

Excel is a widely used and powerful tool for working with data, and it is great for a lot of things. This is convenient and familiar; most of us have had their first experiences with data through Excel or other spreadsheet programs. As Jenny Bryan has said, \href{}{``Excel is how we learn that we love data analysis''}.

Excel is great for data entry. Can also be good for looking at data and feeling like you can touch it, and creating quick exploratory figures.

Excel can also become problematic with extending to analyses. This is because there aren't firm lines between what is data and what is analyses. For example, in this sheet:

\begin{Shaded}
\begin{Highlighting}[]
\NormalTok{knitr}\OperatorTok{::}\KeywordTok{include\_graphics}\NormalTok{(}\StringTok{"img/excel{-}sheet{-}example.png"}\NormalTok{)  }
\CommentTok{\#https://www.smartsheet.com/how{-}to{-}make{-}spreadsheets}
\end{Highlighting}
\end{Shaded}


  \bibliography{book.bib,packages.bib}

\end{document}
