% Options for packages loaded elsewhere
\PassOptionsToPackage{unicode}{hyperref}
\PassOptionsToPackage{hyphens}{url}
%
\documentclass[
]{book}
\usepackage{lmodern}
\usepackage{amssymb,amsmath}
\usepackage{ifxetex,ifluatex}
\ifnum 0\ifxetex 1\fi\ifluatex 1\fi=0 % if pdftex
  \usepackage[T1]{fontenc}
  \usepackage[utf8]{inputenc}
  \usepackage{textcomp} % provide euro and other symbols
\else % if luatex or xetex
  \usepackage{unicode-math}
  \defaultfontfeatures{Scale=MatchLowercase}
  \defaultfontfeatures[\rmfamily]{Ligatures=TeX,Scale=1}
\fi
% Use upquote if available, for straight quotes in verbatim environments
\IfFileExists{upquote.sty}{\usepackage{upquote}}{}
\IfFileExists{microtype.sty}{% use microtype if available
  \usepackage[]{microtype}
  \UseMicrotypeSet[protrusion]{basicmath} % disable protrusion for tt fonts
}{}
\makeatletter
\@ifundefined{KOMAClassName}{% if non-KOMA class
  \IfFileExists{parskip.sty}{%
    \usepackage{parskip}
  }{% else
    \setlength{\parindent}{0pt}
    \setlength{\parskip}{6pt plus 2pt minus 1pt}}
}{% if KOMA class
  \KOMAoptions{parskip=half}}
\makeatother
\usepackage{xcolor}
\IfFileExists{xurl.sty}{\usepackage{xurl}}{} % add URL line breaks if available
\IfFileExists{bookmark.sty}{\usepackage{bookmark}}{\usepackage{hyperref}}
\hypersetup{
  pdftitle={Ocean Health Index Onboarding},
  pdfauthor={OHI Team},
  hidelinks,
  pdfcreator={LaTeX via pandoc}}
\urlstyle{same} % disable monospaced font for URLs
\usepackage{longtable,booktabs}
% Correct order of tables after \paragraph or \subparagraph
\usepackage{etoolbox}
\makeatletter
\patchcmd\longtable{\par}{\if@noskipsec\mbox{}\fi\par}{}{}
\makeatother
% Allow footnotes in longtable head/foot
\IfFileExists{footnotehyper.sty}{\usepackage{footnotehyper}}{\usepackage{footnote}}
\makesavenoteenv{longtable}
\usepackage{graphicx}
\makeatletter
\def\maxwidth{\ifdim\Gin@nat@width>\linewidth\linewidth\else\Gin@nat@width\fi}
\def\maxheight{\ifdim\Gin@nat@height>\textheight\textheight\else\Gin@nat@height\fi}
\makeatother
% Scale images if necessary, so that they will not overflow the page
% margins by default, and it is still possible to overwrite the defaults
% using explicit options in \includegraphics[width, height, ...]{}
\setkeys{Gin}{width=\maxwidth,height=\maxheight,keepaspectratio}
% Set default figure placement to htbp
\makeatletter
\def\fps@figure{htbp}
\makeatother
\setlength{\emergencystretch}{3em} % prevent overfull lines
\providecommand{\tightlist}{%
  \setlength{\itemsep}{0pt}\setlength{\parskip}{0pt}}
\setcounter{secnumdepth}{5}
\usepackage{booktabs}
\usepackage[]{natbib}
\bibliographystyle{apalike}

\title{Ocean Health Index Onboarding}
\author{OHI Team}
\date{2020-10-16}

\begin{document}
\maketitle

{
\setcounter{tocdepth}{1}
\tableofcontents
}
\hypertarget{welcome}{%
\chapter{Welcome}\label{welcome}}

Hello! This book has onboarding resources for the Ocean Health Index (OHI). It is meant as an annotated learning roadmap for groups that are interested in leading your own OHI+ assessment.

\hypertarget{what-is-this-book}{%
\section{What is this book?}\label{what-is-this-book}}

This book aims to help you get ``on board'' with Ocean Health Index (OHI) methods and practices. There are a lot of moving pieces to leading your own OHI assessment - called OHI+ - and it's important to set yours up in a way that is organized and collaborative, expecting that team members might change and that methods might be revisited. So building in resilience and focusing on documentation and sharing from the beginning is critical.

Orient to the book, the websites, ``github repos''
Ohi-science.org incl ohi-global, OHI+ websites.
Oceanhealthindex.org? Website
books

\hypertarget{process}{%
\chapter{Overview of the OHI Process}\label{process}}

\href{https://ohi-science.org/news/three-lessons-global}{Three Lessons for Using the Global Ocean Health Index to Assess Local Oceans}

Orient to the process: Learn \textgreater{} Conduct \textgreater{} etc
PeerJ Fig 1, team roles, seaside chats, have meeting notes, write down decisions, etc. ``structured approach''.
OHI+ examples
Onboarding
transparency, stakeholder engagement, teams open data science (openscapes)

This process is summarized on the \href{https://ohi-science.org/projects/ohi-plus/}{ohi-science.org OHI+ page}.

\hypertarget{transparency}{%
\section{Transparency}\label{transparency}}

\begin{itemize}
\tightlist
\item
  \href{https://ohi-science.org/news/transparent-trust-new-england-seas}{On Building Transparency and Trust for Healthy New England Seas}
\end{itemize}

\hypertarget{open-data-science}{%
\section{Open data science}\label{open-data-science}}

\begin{itemize}
\tightlist
\item
  \href{https://ohi-science.org/news/ohi-and-open-data-science}{Open data science for marine management} (blog)
\item
  \href{https://ohi-science.org/news/scientific-reproducibility-with-fellows}{What does scientific reproducibility look like?} (blog)
\item
  \href{https://ohi-science.org/news/what-is-the-ohi-toolbox}{The OHI Toolbox is useful far beyond ocean management} (blog)
\end{itemize}

\hypertarget{getting-help}{%
\section{Getting help}\label{getting-help}}

\url{https://ohi-science.org/resources/forum/}

\hypertarget{ohi}{%
\chapter{Learn about OHI}\label{ohi}}

OHI is a \href{https://ohi-science.org/news/goal-forward-approach}{goal forward approach}

There is a lot to learn about the Ocean Health Index framework in order to best prepare you for leading your own assessment. We recommend reading and watching the following, to start with a broad overview and then get deeper into different aspects of OHI. Then, continue to come back to review these materials throughout your assessment.

\begin{itemize}
\tightlist
\item
  \url{https://ohi-science.org/learn/}
\end{itemize}

\hypertarget{overviews-and-introductions-to-ohi}{%
\section{Overviews and introductions to OHI}\label{overviews-and-introductions-to-ohi}}

\begin{itemize}
\tightlist
\item
  \href{link}{\textbf{Title}} - (slides). Introduction to OHI framework and global assessments by Ben Halpern and Steve Katona.
\item
  Halpern \& Katona etc
\item
  \href{https://ohi-science.org/news/goal-forward-approach}{\textbf{What makes OHI stand out in the sea of environmental indicators?}}
\end{itemize}

\hypertarget{goals}{%
\section{Goals}\label{goals}}

\begin{itemize}
\item
  NE webinars
\item
  \url{https://ohi-science.org/goals}
\item
  \href{https://ohi-science.org/news/fellowship-expanding-mariculture}{Fellowship Feature: Expanding mariculture} (blog)
\end{itemize}

\hypertarget{plan}{%
\chapter{Plan your assessment}\label{plan}}

Next, you'll need to think about your stakeholders and your team.

\begin{itemize}
\tightlist
\item
  \url{https://ohi-science.org/plan/}
\end{itemize}

\hypertarget{conduct}{%
\chapter{Conduct your assessment}\label{conduct}}

Other training resources are linked from the \href{https://ohi-science.org/training/}{ohi-science.org training page}.

OHI toolbox

\url{http://ohi-science.org/data-science-training/}

\url{https://ohi-science.org/manual/}

The best place to start is with the \href{http://ohi-science.org/toolbox-training/}{OHI Toolbox-training}.

The \href{https://ohi-science.org/manual/}{OHI Manual} is a good reference document as well. (A bit of history: this is how the whole ohi-science.org website started; this was originally a PDF created from a Word Doc that we would email to interested groups).

\hypertarget{inform}{%
\chapter{Inform}\label{inform}}

\url{https://ohi-science.org/inform/}

\hypertarget{global}{%
\chapter{Global assessments}\label{global}}

The OHI team at NCEAS leads global OHI assessments annually; we have iterated and refined our approaches since 2012. . The OHI+ guidance in the previous chapters of this book build from our experiences leading global assessments as well as advising OHI+ assessments. While global assessments are much more complex than most OHI+ assessments, global assessment onboarding is included here for reference.

\url{http://ohi-science.org/ohi-global-guide/index.html}

\begin{itemize}
\tightlist
\item
  populate from syllabi
\end{itemize}

  \bibliography{book.bib,packages.bib}

\end{document}
